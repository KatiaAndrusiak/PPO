~\newline
~\newline
 U\+W\+A\+GA\+: należy użyć doxygen aby wygenerować czytelną instrukcję (zawiera wzory) (aby wygenerować pdf należy w katalogu latex wydać komndę \textquotesingle{}make\textquotesingle{})~\newline


Należy utworzyć brakujące pliki i napisać odpowiednie klasy, opisane niżej. Program musi się kompilować bez ostrzeżeń i poprawnie wykonywać, dając wynik podany w pliku output.\+txt. Najlepiej użyć komendy \textquotesingle{}diff $<$( ./exec ) output.\+txt )\textquotesingle{}

Dodatkowe zastrzeżenia\+:~\newline
a) Należy użyc C\+Make do przygotowania programu\+:~\newline
 -- pliki źrodłowe muszą być w podkatalogu \textquotesingle{}src\textquotesingle{}~\newline
 -- pliki nagłówkowe muszą być w podkatalogu \textquotesingle{}include\textquotesingle{}~\newline
 -- kod musi się kompilować w nowym podkatalogu (np. \textquotesingle{}build\textquotesingle{}) poleceniem \textquotesingle{}cmake .. \&\& make\textquotesingle{} ~\newline
b) program musi się kompilować z flagami \textquotesingle{}-\/\+Wall -\/g\textquotesingle{}, jako minimum.~\newline
c) Jako rozwiązanie, proszę przeslać spakowany katalog źrodlowy, WŁĄ\+C\+Z\+A\+JĄC plik \hyperlink{main_8cpp}{main.\+cpp}, ale B\+EZ katalogu \textquotesingle{}build\textquotesingle{}. ~\newline
d) Nie wolno modyfikować pliku \hyperlink{main_8cpp}{main.\+cpp}.~\newline
e) Proszę nazwać program wykonywalny \textquotesingle{}exec\textquotesingle{}.~\newline






$\ast$$\ast$$\ast$ Klasa \hyperlink{classPolynomial}{Polynomial} $\ast$$\ast$$\ast$

Klasa implementuje wielomian zmiennej rzeczywistej określonego stopnia\+: \[ W_n(x) = a_1 + a_2 x + a_3 x^3 + \dots + a_n x^n \] Minimalne składniki klasy to\+:~\newline
1) dynamiczna tablica współczynników wielomianu.~\newline
 Współczynniki wielomianu $ a_1,\dots,a_n $ są typu double.~\newline
2) Konstruktory, w tym kopiujący, przenoszący.~\newline
 -- Jeden z konstruktorów tworzy wielomian podając stopień wielomianu oraz tablicę współczynników (przykład\+: patrz fcja main).~\newline
 -- Jeden z konstruktorów tworzy wielomian o zadanym stopniu o tablicy współczynników inicjowanej zerami. ~\newline
3) Operatory przypisania kopiującego, przenoszącego~\newline
4) Operator () wyliczający wartość wielomianu dla zadanego argumentu (przykład\+: patrz fcja main)~\newline
5) Operator += dodający wielomiany (przykład\+: patrz fcja main)~\newline
6) Operator $\ast$ mnożący wielomian przez liczbę (przykład\+: patrz fcja main) 7) Operator $<$$<$ drukujący wielomian w formacie widocznym w pliku output.\+txt ~\newline
~\newline




$\ast$$\ast$$\ast$ Klasa Legendre $\ast$$\ast$$\ast$

Klasa pochodna klasy Legendre. Implementuje wielomiany Legendre\textquotesingle{}a.~\newline
Wielomiany Legendre\textquotesingle{}a są rozwiązaniami pewnego równania różniczkowego, i są szeroko stosowane w fizyce i informatyce. ~\newline
Klasa powinna posisadać\+:~\newline
1) metodę obliczającą współczynniki wielomianu Legendre\textquotesingle{}a Współczynnik $ a_l $ wielomianu stopnia $ n $ można otrzymać z następującego bezpośredniego wzoru\+: \[ a_l = \frac{(-1)^{\frac{n-l}{2}}}{2^n} \frac{ (n+l)! } { (\frac{n-l}{2})! (\frac{n+l}{2})! l! } , \] gdzie $ l=0,2,4,\dots $ jeśli $ n $ -\/parzyste oraz $ l=1,3,5,\dots $ jeśli $ n $ -\/nieparzyste. Jako silni można użyć funkcji std\+::tgamma, n! = std\+::tgamma(n+1).~\newline
2) Konstruktory, w tym kopiujący, przenoszący~\newline
 Jeden z konstruktorów ma tworzyć wielomian Legendre\textquotesingle{}a o zadanym stopniu.~\newline
3) Operatory przypisania kopiującego, przenoszącego. 